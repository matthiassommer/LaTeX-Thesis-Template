\chapter{Introduction} 
\label{sec:Introduction}

\section{Motivation}
%Unverfänglicher Satz am Anfang. Anerkannte Fakten zuerst.


%Dann Forschungslücje aufzeigen: drawbacks of current systems, solutions, approaches

%Der letzte Absatz muss klar machen, dass neue Ansätze hermüssen

%übergang zum Thema



%------------
\section{Problem statement \& objectives}
%Hier wird der letzte Absatz aufgegriffen und weitergeführt zu den konkreten Lösungen durch diese Dissertation.
%Wie behandelt die Diss diese Lücke? Welche Lösungen werden aufgezeigt?
%Formulierung der Forschungsfragen und Hypothesen

\epigraph{''Prediction is very difficult, especially if it's about the future.''}{Niels Bohr}
%




%--------
\section{Contributions}




%--------
\section{List of publications}
The work and results presented in this thesis have been published in the following publications:

\paragraph{Book chapters}
\begin{itemize}	
	\item \textit{Matthias Sommer}, Sven Tomforde, and Jörg Hähner: ``An Organic Computing Approach to Resilient Traffic Management''. In \textit{Autonomic Road Transport Support Systems}, pp. 113-130, ISBN 978-3-319-25808-9, 2016.
\end{itemize}

\paragraph{Articles}



\paragraph{Technical reports}


